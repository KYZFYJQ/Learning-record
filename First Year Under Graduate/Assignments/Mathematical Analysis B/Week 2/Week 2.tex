\documentclass[12pt, a4paper, oneside]{ctexart}
\usepackage{amsmath, amsthm, amssymb, bm, color, framed, graphicx, hyperref, mathrsfs}

\title{\textbf{数学分析B 2025秋 USTC}}
\author{姓名:石泊远$ \hspace{1cm} $学号:PB25000051}
\date{\today}
\linespread{1.5}
\definecolor{shadecolor}{RGB}{241, 241, 255}
\newcounter{problemname}
\newenvironment{problem}{\begin{shaded}\stepcounter{problemname}\par\noindent\textbf{Assignments ~ \arabic{problemname}. }}{\end{shaded}\par}
\newenvironment{solution}{\par\noindent\textbf{Proof. }}{\par}
\newenvironment{note}{\par\noindent\textbf{Assignments ~ \arabic{problemname}'s Remark. }}{\par}
\newenvironment{remark}{\noindent \textbf{Remark.}}{}

\begin{document}
	%Introduction
	\maketitle
	
	%main body
	\begin{problem}
		设$r,s \in \mathbb{Q}$,求证:若$r+s \sqrt{2} = 0$,则$r=s=0$
	\end{problem}
	
	\begin{solution}
		若$r,s$中有一个不等于0,不妨设为$s$,$r \neq 0$的情况可以通过同乘$\sqrt{2}$转化到$ s \neq 0 $的情况 \\
		则原式等价于$\sqrt{2} = -\dfrac{r}{s} \in \mathbb{Q}$,而我们知道$\sqrt{2}$不是有理数,从而矛盾,有两个不等于$0$同理 \\
	\end{solution}
	
	\begin{problem}
		证明:$6 \mid f(n) = n^4+2n^3+2n^2+n$
	\end{problem}
	
	\begin{solution}
		$$f(n) = n(n+1)(n^2+n+1)$$
		相邻两个数中定有一个偶数,从而$2 \mid f(n)$ \\
		把模三的完全剩余系带入可知$3 \mid f(n)$
	\end{solution}
	
	\begin{problem}
		验证$E = \left\{ p+q\sqrt{2} \mid p,q \in \mathbb{Q} \right\}$是一个数域,并证明$\mathbb{Q} \subsetneqq \mathbb{E} \subsetneqq \mathbb{R} $。 \\
		问:$\mathbb{Q},\mathbb{E},\mathbb{R}$之间还有没有其他的数域$E_1$
	\end{problem}
	
	\begin{solution}
		$E$的关于加减乘的封闭性是显然的,除的封闭性可以由如下被证明$$ \dfrac{p+q\sqrt{2}}{s+t\sqrt{2}} = \dfrac{ps-2qt+(qs-pt)\sqrt{2}}{s^2 - 2t^2} \in \mathbb{E}$$
		$\mathbb{Q} \subset \mathbb{E} \subset \mathbb{R}$是显然的,下面证明真包含 \\
		$\sqrt{2} \in \mathbb{E},\sqrt{2} \notin \mathbb{Q} \Longrightarrow \mathbb{Q} \subsetneqq \mathbb{E}$ \\
		若$1,\sqrt{2}$可以通过线性组合表出$\sqrt{3}$,不妨设$\sqrt{3} = r+s\sqrt{2} \Longleftarrow \dfrac{r^2+2s^2-3}{2rs} = \sqrt{2}$,矛盾,从而我们有\\
		$\sqrt{3} \in \mathbb{R},\sqrt{3} \notin \mathbb{E} \Longrightarrow \mathbb{E} \subsetneqq \mathbb{R}$ \\
	\end{solution}
\end{document}